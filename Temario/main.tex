\documentclass[12pt,letterpaper]{article}
\usepackage{graphicx}
\usepackage{scrextend}
\usepackage{vmargin}
\usepackage[utf8]{inputenc}
\usepackage[spanish]{babel}
\usepackage{subcaption}
\usepackage{caption}
\usepackage{multicol}
\usepackage{hyperref}
\usepackage{amsmath, amsthm, amssymb, amsfonts}
\usepackage[usenames]{color}
\usepackage{float}
\parindent=0mm
\pagestyle{empty}
\definecolor{citecolor}{rgb}{.12,.54,.11}
\definecolor{urlcolor}{RGB}{43, 137, 30 }
\definecolor{1ee592}{RGB}{30,229,146}
\definecolor{1bac70}{RGB}{27,172,112}
\definecolor{f38638}{RGB}{243,134,56}
\definecolor{black}{RGB}{0,0,0}
\definecolor{gray}{RGB}{156,156,156}
\hypersetup{
    colorlinks=true,
    linkcolor=blue,
    filecolor=magenta,      
    urlcolor=urlcolor,
    citecolor=citecolor,
}
\title{Temario}
\author{giovannilopez9808 }
\date{January 2021}

\begin{document}
\begin{figure}[H]
    \begin{flushright}
        \includegraphics[scale=0.25]{../images/logo.png}
    \end{flushright}
\end{figure}
\vspace{-5cm}

\begin{enumerate}
    \item Principios de git / control de versiones
    \item Uso de google colab
    \item Tipo de datos, variables y como usarlas
    \begin{enumerate}
        \item Booleano
        \item Float
        \item integrales
        \item String
        \item Diccionarios, listas y tuplas
    \end{enumerate}
    \item If y else
    \item Ciclos, for y while
    \begin{enumerate}
        \item Contadores
        \item Iterativos
    \end{enumerate}
    \item Funciones
    \item Numpy, pandas
    \begin{enumerate}
        \item Operaciones con matrices
        \item Primer proyecto - Programar el \href{https://en.wikipedia.org/wiki/Conway%27s_Game_of_Life}{Juego de la vida}
        \item Lectura y escritura de datos
    \end{enumerate}
    \item Matplotlib
    \begin{enumerate}
        \item Scatter, plot
        \item Mapas de colores/imshow
        \item Leyendas con latex
        \item Segundo proyecto - \href{https://en.wikipedia.org/wiki/Edge_detection}{Detección de bordes}
    \end{enumerate}
    \item Problemas matemáticos de resueltos de manera numérica - Scipy
    \begin{enumerate}
        \item Resolver integrales
        \item Resolver ecuaciones diferencial
        \item Tercer proyecto - \href{https://en.wikipedia.org/wiki/Double_pendulum}{Péndulo doble}
        \item Obtener eigenvalores y eigenvectores
        \item Estadística descriptiva
        \item Regresión lineal
        \item Cuarto proyecto - \href{https://es.wikipedia.org/wiki/Potencial_de_Lennard-Jones}{Simulaciones con el potencial de Lennard Jones}
    \end{enumerate}
    \item Problemas avanzados
    \begin{enumerate}
        \item Librería de \href{https://www.youtube.com/channel/UCYO_jab_esuFRV4b17AJtAw}{3Blue1Brown}
        \item \href{https://es.wikipedia.org/wiki/Inteligencia_artificial}{Inteligencia artificial}
        \item \href{https://es.wikipedia.org/wiki/Teleportaci%C3%B3n_cu%C3%A1ntica}{Teleportación cuántica}
    \end{enumerate}
    \item Proyecto final - optional
\end{enumerate}
\end{document}